\documentclass{article}

% Pacotes %%%%%%%%%%%%%%%%%%%%%%%%% {{{
% Pacote para margem %%%%%%%%%%%%%% {{{
% Ajusta a margem do documento
\usepackage[margin=3.5cm]{geometry}
%%%%%%%%%%%%%%%%%%%%%%%%%%%%%%%%%%% }}}

% Pacote para URL %%%%%%%%%%%%%%%%% {{{
\usepackage{hyperref}
\hypersetup{colorlinks=true,linkcolor=blue,filecolor=magenta,urlcolor=cyan,}
%%%%%%%%%%%%%%%%%%%%%%%%%%%%%%%%%%% }}}

% Pacote de destaque de sintaxe %%% {{{
% Corrige o diretório de saída do minted
\usepackage[abspath]{currfile}
\getabspath{\jobname.log}
% Carrega o minted (para destaque de sintaxe Coq)
\usepackage[newfloat, outputdir=\theabsdir]{minted}
% Configura ambiente referenciável
\usepackage{caption}
\newenvironment{codigo}{\captionsetup{type=listing}}{}
\SetupFloatingEnvironment{listing}{name=Trecho de código}
%%%%%%%%%%%%%%%%%%%%%%%%%%%%%%%%%%% }}}

% Pacote de data %%%%%%%%%%%%%%%%%% {{{
\usepackage[portuguese]{babel}
\usepackage[useregional]{datetime2}
%%%%%%%%%%%%%%%%%%%%%%%%%%%%%%%%%%% }}}

% Outros pacotes %%%%%%%%%%%%%%%%%% {{{
\usepackage{color}
\usepackage{amsthm}
\usepackage{mathtools}
\usepackage[utf8]{inputenc}
%%%%%%%%%%%%%%%%%%%%%%%%%%%%%%%%%%% }}}

% Custom commands %%%%%%%%%%%%%%%%% {{{
\newcommand{\nantes}[1]{\textcolor{blue}{#1}}
%%%%%%%%%%%%%%%%%%%%%%%%%%%%%%%%%%% }}}

% Macros de teoremas %%%%%%%%%%%%%% {{{
\newtheorem{definicao}{Definição}
\newtheorem{exemplo}{Exemplo}
\newtheorem{lema}{Lemma}
\newtheorem{observacao}{Observação}
\newtheorem{teorema}{Teorema}

\newtheorem{defi}{Definição}
\newtheorem{exam}{Exemplo}
\newtheorem{lem}{Lemma}
\newtheorem{obs}{Observação}
\newtheorem{teo}{Teorema}
%%%%%%%%%%%%%%%%%%%%%%%%%%%%%%%%%%% }}}

% Título %%%%%%%%%%%%%%%%%%%%%%%%%% {{{
\title{Definição de uma gramática de termos usando Coq}
\author{Vinícius T M Sugimoto}
\date{\today}
%%%%%%%%%%%%%%%%%%%%%%%%%%%%%%%%%%% }}}
%%%%%%%%%%%%%%%%%%%%%%%%%%%%%%%%%%% }}}

\begin{document}

\maketitle

\section{Introdução} % {{{
\label{section:introducao}

A formalização de uma gramática é um passo essencial para o estudo de problemas complexos.
Uma vez que uma gramática está bem definida, ela ajuda a fazer com que o progresso no trabalho
em alguma determinado problema também esteja bem definido e correto. No entanto, a definição
de uma gramática por si só pode ser considerada um problema, que também precisa ser bem definido.
Assim, existem ferramentas que auxiliam na formalização de provas de diversos problemas, que por
sua vez já estão corretos e garantem que seus resultados também estejam.

A formalização de uma gramática de um problema em termos de uma linguagem bem estruturada é
extremamente importante pois permite com que futuras definições a partir dessa gramática utilizem
resultados provados previamente em outros momentos e por outras pessoas. A título de exemplo, pode-se
considerar a construção dos números naturais a partir dos axiomas de Peano. Sem uma definição clara
desses números é difícil dizer quais resultados são válidos ou não, por outro lado, com uma definição
bem estruturada podemos utilizar resultados obtidos sobre esses números há séculos. Neste trabalho
apresenta-se a definição de uma gramática simples de termos e funções, que por sua vez poderá ser
utilizada para estudar problemas mais complexos e interessantes.

% Segundo

% Terceiro

Para a construção da gramática proposta neste trabalho, precisa-se que seja possível mostrar que essa
construção está correta. Por um lado este trabalho poderia tentar provar a corretude de tudo do que a
gramática a ser apresentada depende, assim concluindo a corretude deste trabalho também. Mas quanto
mais este trabalho tem que provar maior a chance de um erro acontecer ou da qualidade das provas não
serem como desejadas. Assim, a melhor escolha é utilizar os resultados cuidadadosamente provados por
outros trabalhos. Mais especificamente, neste trabalho será utilizado o programa Assistente de Provas
Coq e diversos resultados provados e bem definidos que o Coq traz.

Na seções seguintes, \ref{section:a_gramatica} e \ref{section:assistente_coq}, será mostrado em maiores
detalhes a forma da gramática que será construída e a maneira que foi definida utilizando o assistente
de provas Coq, respectivamente. Depois, na seção \ref{section:resultados}, serão apresentados os
resultados provados obtidos sobre a definição da gramática.
% }}}

\section{A Gramática} % {{{
\label{section:a_gramatica}

Este trabalho constrói uma gramática de termos recursiva com quatro tipos de termos. Esses termos tem
tentam representar a definição de funções, com aridades possivelmente infinitas. Assim, os termos desta
gramática podem ser: uma Variável, que são representadas por números naturais e representam argumentos
e parâmetros de funções, termos Unitários, que encapsulam um outro termo da gramática, termos Pares,
que encapsulam dois outros termos da gramática, e Funções, que representam funções sobre variáveis e
são representadas com um natural, sua aridade, e um termo, que expande para seus argumentos ou
parâmetros, em forma de variáveis. A gramática pode ser visualizada melhor na seguinte notação:

\begin{equation}
	\begin{split}
		\langle NATURAL \rangle  & := 0 | 1 | 2 | ... \\
		\langle TERMO \rangle    & :=
			\langle VARIAVEL \rangle |
			\langle UNITARIO \rangle |
			\langle PAR \rangle |
			\langle FUNCAO \rangle \\
		\langle VARIAVEL \rangle & := \langle NATURAL \rangle \\
		\langle UNITARIO \rangle & := \langle TERMO \rangle \\
		\langle PAR \rangle      & := \langle TERMO \rangle \langle TERMO \rangle \\
		\langle FUNCAO \rangle   & := \langle NATURAL \rangle \langle TERMO \rangle
	\end{split}
\end{equation}

Para um melhor entendimento, uma leitura dessa definição em linguagem mais natural pode ser dada como
na definição \ref{definicao:pre-termos} e alguns exemplos de termos desta gramática são mostrados no
exemplo \ref{exemplo:pre-termos}.

\begin{definicao}[Pré-termos]
	Seja $\Sigma=\{\texttt{Un, Pr, F}\}$ um conjunto finito de símbolos tais que $\texttt{Un}$ é
	unário, $\texttt{Pr}$ é binário e $\texttt{F}$ é um símbolo de função de aridade variável, e $X$ um
	conjunto de variáveis tal que $X \cap \Sigma = \emptyset$.
	O conjunto dos termos $T(\Sigma, X)$ construído a partir de $\Sigma$ e $X$, pode definido indutivo
	da seguinte maneira:
	\begin{itemize}
		\item $X \subset T(\Sigma, X)$, i.e., toda variável é um termo;
		\item Se $t$ é um termo, então $\texttt{Un}(t)$ é um termo;
		\item Se $t_1$ e $t_2$ são  termos, então $\texttt{Pr}(t_1, t_2)$ é um termo;
	 	\item Se $t_1, t_2, \ldots, t_n$ são  termos, então $\texttt{F}(t_1,t_2,\ldots, t_n)$ é um
			 termo, para todo $n \in \mathbf{N}^+$.
	\end{itemize}
	\label{definicao:pre-termos}
\end{definicao}

\begin{exemplo}
	Note que para variáveis $x_1,x_2,x_3$ de $X$ temos os seguintes termos:
	\begin{itemize}
		\item $x_1, x_2, x_3$ são termos;
		\item $\texttt{F}(x_1)$,$\texttt{F}(x_1, x_2)$, $\texttt{F}(x_1, x_2, x_3, x_1)$, etc.
	\end{itemize}
	\label{exemplo:pre-termos}
\end{exemplo}

A definição de termos permite que a função $F$ seja aplicada a uma sequência de termos com tamanho
variável, mas finito.
Essa gramática com esta definição, apesar de poder ser bem definida, não é tão versátil. Por exemplo,
os termos de funções da gramática, apesar de terem um natural associado para indicar sua aridade, nada
impede que o termo associado não tenha essa aridade. Ainda sobre as funções, elas, na definição
apresentada, são limitadas a uma função para cada diferente aridade, isso pela falta de um
identificador associado às funções. Ainda assim, para um primeiro momento, o trabalho foi realizado
relevando os problemas apresentados e, por isso, os termos dessa gramática serão chamados daqui em
diante de pré-termos.

Na seção seguinte é apresentado como está gramática é definida no Assistente de Provas Coq bem como como
algumas propriedades desta gramática são provadas.

% }}}

\section{Assistente de Provas Coq} % {{{
\label{section:assistente_coq}

O assistente de provas Coq é um programa que auxilia o desenvolvimento de provas formais. Suas
aplicações comuns incluem a certificação de propriedades de linguagens de programação, % cite coq inria
formalização da matemática e ensino. Neste trabalho o Coq é utilizado para auxiliar na formalização da
definição da gramática deste trabalho e na prova de suas propriedades.

O programa é gratuito e pode ser obtido em diversas versões no site oficial
\url{https://coq.inria.fr/}. Para este trabalho foi utilizada a versão 8.12 do Coq, tendo sido
compilada a partir do fonte em um computador com Linux Ubuntu 20.04 64-bits.

Com a ferramenta Coq, a definição da gramática apresentada na seção \ref{section:a_gramatica} é
dividida em duas partes e tem uma sintaxe diferente, própria do Coq. O trecho de código
\ref{codigo:pre-termos} a seguir mostra como um gramática recursiva equivalente à mostrada na
definição \ref{definicao:pre-termos} é construída.

\begin{codigo}
\begin{minted}{coq}
	Inductive Var : Set := var : nat -> Var.
	Inductive pterm : Type :=
	    V : Var -> pterm
	  | Un: pterm -> pterm
	  | Pr: pterm -> pterm -> pterm
	  | F : nat -> pterm -> pterm
	.
\end{minted}
\caption{Definição do pterm}
\label{codigo:pre-termos}
\end{codigo}

Com o auxílio do Coq e a definição no formato de \ref{codigo:pre-termos}, pode-se provar propriedades
sobre a gramática definidida. O Coq permite a definição de funções sobre o tipo de dados definido pelos
pré-termos e prova formal de lemas e teoremas usando táticas de provas já definidas em bibliotecas
padrão, em bibliotecas externas ou pelo próprio usuário. O formato de uma definição de uma função no
Coq é como demonstrado no trecho de código em \ref{codigo:funcao}. Como a gramática de pré-termos
definida neste trabalho representa um tipo indutivo, a definição de funções recursivas também é
possível pelo Coq, sendo essas bastante utilizadas neste trabalho. O formato de uma definição de uma
função recursiva no Coq é como demonstrado no trecho de código em \ref{codigo:funcao_recursiva}. 

\begin{codigo}
\begin{minted}{coq}
	Definition fun1 (parâmetros: Tipo): Tipo :=
	  (* corpo da função *).
\end{minted}
\caption{Modelo de função}
\label{codigo:funcao}
\end{codigo}

\begin{codigo}
\begin{minted}{coq}
	Fixpoint fun2 (parâmetros: Tipo): Tipo :=
	  (*
	  corpo da função com chamada
	  recursiva sobre o subproduto
	  de algum parâmetro definido
	  *).
\end{minted}
\caption{Modelo de função recursiva}
\label{codigo:funcao_recursiva}
\end{codigo}

Com a linguagem Coq, pode-se, então, declarar funções sobre a gramática de pré-termos ou outros tipos
de dados como demonstrado no exemplo \ref{exemplo:funcoes_pre-termos}, em que são demonstradas as
funções \texttt{soma\_ab} e \texttt{soma\_ate\_n}, respectivamente. As funções apresentadas no exemplo
\ref{exemplo:funcoes_pre-termos} são simples, mas exemplificam a estrutura sintática de uma definição.
Uma vez que a sintaxe da linguagem se torne familiar, funções mais complexas e úteis podem ser
implementadas.

\begin{exemplo}
	\label{exemplo:funcoes_pre-termos}
\end{exemplo}
\begin{codigo}
\begin{minted}{coq}
	Definition soma_ab (a b: nat): nat := a + b.
\end{minted}
\caption{Função que dois naturais $a$ e $b$}
\label{codigo:soma_ab}
\end{codigo}

\begin{codigo}
\begin{minted}{coq}
	Fixpoint soma_ate_n (n: nat): nat :=
	  match n with
		0 => 0
		S p => n + soma_ate_n p
		(* chamada recursiva sobre o subproduto de n = p *)
	  end.
\end{minted}
\caption{Função que soma todos os naturais até $n$}
\label{codigo:soma_ate_n}
\end{codigo}

O Coq também oferece funcionalidades para provar propriedades sobre diferentes tipos de dados, como,
por exemplo, os pré-termos definidos neste trabalho. A sintaxe para esses tipos de provas são como
demonstrado nos trechos de código em \ref{codigo:provas}. No exemplo
\ref{exemplo:decidibilidade_naturais}, é apresentada a prova da decidibidade da igualdade dos naturais
usando uma tática padrão do Coq, o que mostra o quão poderoso pode ser o auxílio do Coq para se fazer
provas formais.

\begin{codigo}
\begin{minted}{coq}
	Definition def1: (* definição *).
	Proof.
	  (* passos da prova *)
	Defined.

	Lemma lemma1: (* lema *).
	Proof.
	  (* passos da prova *)
	Qed.

	Theorem theo1: (* teorema *).
	Proof.
	  (* passos da prova *)
	Qed.
\end{minted}
\caption{Definições, lemas e teoremas em Coq}
\label{codigo:provas}
\end{codigo}

\begin{exemplo}
	\label{exemplo:decidibilidade_naturais}
\end{exemplo}
\begin{codigo}
\begin{minted}{coq}
	Lemma nat_eqdec: forall (m n: nat), {m = n} + {m <> n}.
	Proof.
	  decide equality.
	Defined.
\end{minted}
\caption{Prova da decidibilidade da igualdade em naturais}
\label{codigo:decidibilidade_naturais}
\end{codigo}

A tática utilizada, \texttt{decide equality}, é poderosa o suficiente para terminar a prova por si só.
Provas como esta não são apenas exemplos de como o Coq pode ser utilizado, mas também podem ser
utilizadas em problemas e trabalhos reais, de fato, esta é uma prova que é utilizada neste trabalho
para auxiliar na construção da demonstração da decidibilidade da igualdade para os pré-termos
definidos. Na seção seguinte serão apresentados as propriedades e provas feitas sobre os pré-termos com
o Coq e como o progresso do trabalho foi feito até chegar aos resultados finais.
% }}}

\section{Resultados} % {{{
\label{section:resultados}

Neste trabalho, uma gramática de pré-termos foi definida (\ref{section:a_gramatica}), mas o seu
objetivo é demonstrar propriedades sobre essa gramática. Mais especificamente, este trabalho foca e
demonstrar propriedades sobre os pré-termos de tipo \texttt{F}, que representam funções sobre outros
pré-termos. Nesta seção serão apresentadas funções definidas sobre pré-termos e em seguida serão
apresentados propriedades que foram provadas sobre pré-termos para a definição dessas funções.

Os resultados finais alcançados neste projeto são funções sobre que computam subtermos de um pré-termo,
a quantidade de parâmetros que um pré-termo \texttt{F} está recebendo, se um pré-termo \texttt{F}
representa uma função bem formada, a quantidade de ocorrências de um pré-termo em outro pré-termo etc.
Na definição apresentadas a seguir (\ref{definicao:parameter}, \ref{definicao:is_F}, \ref{definicao:count_var}, \ref{definicao:psubterm}, \ref{definicao:poccur}, \ref{definicao:plength}, \ref{definicao:pterm_iarg}, \ref{definicao:pterm_ith}), as assinaturas dessas funções e de outras são apresentadas junto a uma definição de suas funcionalidades.

\begin{definicao}[parameter]
	Seja $parameter: pterm \rightarrow nat$ uma função que computa a quantidade de parâmetros que um
	pré-termo recebe, se o termo raiz desse pré-termo é um \texttt{F}. A assinatura de
	\texttt{parameter} como uma função recursiva é:

	\texttt{Fixpoint parameter (t: pterm): nat}
	\label{definicao:parameter}
\end{definicao}

\begin{definicao}[is\_F]
	Seja $is\_F: pterm \rightarrow bool$ uma função que computa se um pré-termo $t$ representa uma
	função bem formada, isto é, se $t$ é um \texttt{F} e sua aridade $n$ é igual a $parameter(t)$. A
	assinatura de \texttt{is\_F} como uma função recursiva é:

	\texttt{Fixpoint is\_F (t: pterm): bool}
	\label{definicao:is_F}
\end{definicao}

\begin{definicao}[count\_var]
	Seja $count\_var: pterm \rightarrow nat$ uma função que computa a quantidade de pré-termos
	\texttt{V} em um pré-termo. A assinatura de \texttt{count\_var} como uma função recursiva é:

	\texttt{Fixpoint count\_var (t: pterm): nat}
	\label{definicao:count_var}
\end{definicao}

\begin{definicao}[psubterm]
	Seja $psubterm: pterm \rightarrow \mathcal{P}(pterm)$, em que $\mathcal{P}(pterm)$ é o conjunto das
	partes de pterm, uma função que computa os subtermos de um pré-termo. A assinatura de
	\texttt{psubterm} como uma função recursiva é:

	\texttt{Fixpoint psubterm (t: pterm): list pterm}
	\label{definicao:psubterm}
\end{definicao}

\begin{definicao}[poccur]
	Seja $poccur: pterm \times pterm \rightarrow nat$ uma função que computa a quantidade de
	ocorrências de um pré-termo $i$ em um pré-termo $t$. A assinatura de \texttt{poccur} como uma
	função recursiva é:

	\texttt{Fixpoint poccur (i t: pterm): nat}
	\label{definicao:poccur}
\end{definicao}

\begin{definicao}[plength]
	Seja $plength: pterm \rightarrow nat$ uma função que computa o comprimento de um pré-termo, isto é,
	dada uma representação em árvore de um pré-termo, como na definição
	\ref{definicao:pre-termo_arvore}, a quantidade de nós dessa árvore. A assinatura de
	\texttt{plength} como uma função recursiva é:

	\texttt{Fixpoint plength (t: pterm): nat}
	\label{definicao:plength}
\end{definicao}

\begin{definicao}[pterm\_iarg]
	Seja $pterm\_iarg: pterm \times nat -> pterm$ uma função que computa o $i$-ésimo argumento de um
	pré-termo $t$, se $t$ é um \texttt{F}. Se $i$ é um índice inválido a função computa $\emptyset$. A
	assinatura de \texttt{pterm\_iarg} como uma função recursiva é:

	\texttt{Fixpoint pterm\_iarg (t: pterm) (i: nat): option pterm}
	\label{definicao:pterm_iarg}
\end{definicao}

\begin{definicao}[pterm\_ith]
	Seja $pterm\_ith: pterm \times nat \rightarrow pterm$ uma função que computa o $i$-ésimo termo de
	um pré-termo, isto é, dada a representação em árvore de um pré-termo, como dada na definição
	\ref{definicao:pre-termo_arvore}, computa o $i$-ésimo nó da árvore em uma busca em largura. A
	assinatura de $pterm\_ith$ é: 

	\texttt{Definition pterm\_ith (t: pterm) (i: nat): option pterm}
	\label{definicao:pterm_ith}
\end{definicao}

\begin{definicao}[Representação em árvore de um pré-termo]
	\label{definicao:pre-termo_arvore}
\end{definicao}

% }}}

\end{document}
