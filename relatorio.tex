\documentclass{article}
\usepackage[utf8]{inputenc}

\usepackage[margin=3.5cm]{geometry}

\usepackage{hyperref,color,amsthm}
\hypersetup{
	colorlinks=true,
	linkcolor=blue,
	filecolor=magenta,
	urlcolor=cyan,
}
\usepackage{mathtools}
\usepackage{minted}
\newcommand{\nantes}[1]{\textcolor{blue}{#1}}
\newcommand{\tty}[1]{{\tt #1}}

%%% macros para teoremas %%
\newtheorem{defi}{Definição}
\newtheorem{exam}{Exemplo}
\newtheorem{lem}{Lemma}
\newtheorem{obs}{Observação}
\newtheorem{teo}{Teorema}

\title{Definição de uma gramática de termos usando Coq}
\author{Vinícius T M Sugimoto}
\date{Dezembro 2020}

\begin{document}

\maketitle

\section{Introdução}
\label{section:introducao}


A formalização de uma gramática é um passo essencial para o estudo de problemas complexos.
Uma vez que uma gramática está bem definida, ela ajuda a fazer com que o progresso no trabalho
em alguma determinado problema também esteja bem definido e correto. No entanto, a definição
de uma gramática por si só pode ser considerada um problema, que também precisa ser bem definido.
Assim, existem ferramentas que auxiliam na formalização de provas de diversos problemas, que por
sua vez já estão corretos e garantem que seus resultados também estejam.

A formalização de uma gramática de um problema em termos de uma linguagem bem estruturada é
extremamente importante pois permite com que futuras definições a partir dessa gramática utilizem
resultados provados previamente em outros momentos e por outras pessoas. A título de exemplo, pode-se
considerar a construção dos números naturais a partir dos axiomas de Peano. Sem uma definição clara
desses números é difícil dizer quais resultados são válidos ou não, por outro lado, com uma definição
bem estruturada podemos utilizar resultados obtidos sobre esses números há séculos. Neste trabalho
apresenta-se a definição de uma gramática simples de termos e funções, que por sua vez poderá ser
utilizada para estudar problemas mais complexos e interessantes.

% Segundo

% Terceiro

Para a construção da gramática proposta neste trabalho, precisa-se que seja possível mostrar que essa
construção está correta. Por um lado este trabalho poderia tentar provar a corretude de tudo do que a
gramática a ser apresentada depende, assim concluindo a corretude deste trabalho também. Mas quanto
mais este trabalho tem que provar maior a chance de um erro acontecer ou da qualidade das provas não
serem como desejadas. Assim, a melhor escolha é utilizar os resultados cuidadadosamente provados por
outros trabalhos. Mais especificamente, neste trabalho será utilizado o programa Assistente de Provas
Coq e diversos resultados provados e bem definidos que o Coq traz.

Na seções seguintes, \ref{section:a_gramatica} e \ref{section:assistente_coq}, será mostrado em maiores
detalhes a forma da gramática que será construída e a maneira que foi definida utilizando o assistente de provas Coq, respectivamente. Depois, na seção \ref{section:resultados}, serão apresentados os
resultados provados obtidos sobre a definição da gramática.

\section{A Gramática}
\label{section:a_gramatica}

Este trabalho constrói uma gramática de termos recursiva com quatro tipos de termos. Esses termos tem  tentam representar a definição de funções, com aridades possivelmente infinitas. Assim, os termos desta
gramática podem ser: uma Variável, que são representadas por números naturais e representam argumentos
e parâmetros de funções, termos Unitários, que encapsulam um outro termo da gramática, termos Pares,
que encapsulam dois outros termos da gramática, e Funções, que representam funções sobre variáveis e
são representadas com um natural, sua aridade, e um termo, que expande para seus argumentos ou
parâmetros, em forma de variáveis. A gramática pode ser visualizada melhor na seguinte notação:


\begin{equation}
	\begin{split}
		\langle NATURAL \rangle  & := 0 | 1 | 2 | ... \\
		\langle TERMO \rangle    & :=
			\langle VARIAVEL \rangle |
			\langle UNITARIO \rangle |
			\langle PAR \rangle |
			\langle FUNCAO \rangle \\
		\langle VARIAVEL \rangle & := \langle NATURAL \rangle \\
		\langle UNITARIO \rangle & := \langle TERMO \rangle \\
		\langle PAR \rangle      & := \langle TERMO \rangle \langle TERMO \rangle \\
		\langle FUNCAO \rangle   & := \langle NATURAL \rangle \langle TERMO \rangle
	\end{split}
\end{equation}

Essa gramática com esta definição, apesar de poder ser bem definida, não é tão versátil. Por exemplo,
os termos de funções da gramática, apesar de terem um natural associado para indicar sua aridade, nada
impede que o termo associado não tenha essa aridade. Ainda sobre as funções, elas, na definição
apresentada, são limitadas a uma função para cada diferente aridade, isso pela falta de um
identificador associado às funções. Ainda assim, para um primeiro momento, o trabalho foi realizado
relevando os problemas apresentados.

Na seção seguinte é apresentado como está gramática é definida no Assistente de Provas Coq bem como como algumas propriedades desta gramática são provadas.

\nantes{
\begin{defi}[Pré-Termos]
Sejam $\Sigma=\{\tty{Un},{\tt Pr},{\tt F}\}$ um conjunto finito de símbolos tais que $\tty{Un}$ é unário, $\tty{Pr}$ é binário e $\tty{F}$ é um símbolo de função de aridade variável, e $X$ um conjunto de variáveis tal que $X\cap \Sigma=\emptyset$. O conjunto dos pré-termos $T(\Sigma,X)$ construído a partir de $\Sigma$ e $X$, pode definido indutivo da seguinte maneira:
\begin{itemize}
    \item $X\subset T(\Sigma,X)$, i.e., toda variável é um pré-termo;
    \item Se $t$ é um pré-termo, então $\tty{Un}(t)$ é um pré-termo;
    \item Se $t_1$ e $t_2$ são  pré-termos, então $\tty{Pr}(t_1,t_2)$ é um pré-termo;
     \item Se $t_1, t_2, \ldots, t_n$ são  pré-termos, então $\tty{F}(t_1,t_2,\ldots, t_n)$ é um pré-termo, para todo $n\in \mathbf{N}^+$.
\end{itemize}
\end{defi}
}


\nantes{
\begin{exam}
Note que para variáveis $x_1,x_2,x_3$ de $X$ temos os seguintes pré-termos:
\begin{itemize}
    \item $x_1,x_2,x_3$ são pré-termos;
    \item $\tty{F}(x_1)$,$\tty{F}(x_1,x_2)$, $\tty{F}(x_1,x_2,x_3,x_1)$, etc.
\end{itemize}
A definição de pré-termos permite que a função $F$ seja aplicada a uma sequência de termos com tamanho variável, mas finito.
\end{exam}
}
\section{Assistente de Provas Coq}
\label{section:assistente_coq}

O assistente de provas Coq é um programa que auxilia o desenvolvimento de provas formais. Suas
aplicações comuns incluem a certificação de propriedades de linguagens de programação, % cite coq inria
formalização da matemática e ensino. Neste trabalho o Coq é utilizado para auxiliar na formalização da
definição da gramática deste trabalho e na prova de suas propriedades.

O programa é gratuito e pode ser obtido em diversas versões no site oficial
\url{https://coq.inria.fr/}. Para este trabalho foi utilizada a versão 8.12 do Coq, tendo sido
compilada a partir do fonte em um computador com Linux Ubuntu 20.04 64-bits.

Com a ferramenta Coq, a definição da gramática apresentada na seção \ref{section:a_gramatica} é
dividida em duas partes e tem uma sintaxe diferente, própria do Coq. O trecho de código Coq a seguir
mostra como um gramática recursiva equivalente à mostrada na seção \ref{section:a_gramatica} é
construída.

\begin{minted}{coq}
	Inductive Var : Set := var : nat -> Var.

	Inductive pterm : Type :=
		V : Var -> pterm
	  | Un: pterm -> pterm
	  | Pr: pterm -> pterm -> pterm
	  | F : nat -> pterm -> pterm
	.
\end{minted}

\section{Resultados}
\label{section:resultados}

\end{document}




